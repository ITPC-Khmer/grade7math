\section*{ចម្លើយ (សម្រាប់គ្រូ)}

\begin{multicols}{2}
\subsection*{ក. តម្លៃខ្ទង់}
\begin{enumerate}[label=\arabic*.]
    \item $4\times10^4 + 7\times10^2 + 6$
    \item ខ្ទង់ពាន់គឺ 8
    \item $80\,632$
\end{enumerate}

\subsection*{ខ. ការប្រៀបធៀប}
\begin{enumerate}[label=\arabic*.]
    \item $504\,080 > 503\,890$
    \item $9\,580 < 9\,805 < 9\,850$
    \item ស្មើគ្នា ($99\,999$)
\end{enumerate}

\subsection*{គ. ប្រមាណវិធី}
\begin{enumerate}[label=\arabic*.]
    \item $9\,242$
    \item $4\,144$
    \item $7\,600$
    \item $546$
\end{enumerate}

\subsection*{ឃ. កត្តា និងចំនួនបឋម}
\begin{enumerate}[label=\arabic*.]
    \item $1, 2, 3, 4, 6, 9, 12, 18, 36$
    \item បឋម (គ្មានកត្តាចែក)
    \item $\mathrm{LCM}(12,18)=36$
    \item $\mathrm{GCF}(24,30)=6$
\end{enumerate}

\subsection*{ង. វិធានចែកដាច់}
\begin{enumerate}[label=\arabic*.]
    \item ដាច់នឹង 3, មិនដាច់នឹង 9
    \item $2, 5, 10$
\end{enumerate}

\subsection*{បញ្ចប់មេរៀន}
\begin{enumerate}[label=\arabic*.]
    \item ប្រាំពីររយប្រាំពាន់ហុកសិប
    \item មិនបឋម ($11\times11$)
\end{enumerate}
\end{multicols}

\vfill
\noindent\textit{កំណត់ចំណាំសម្រាប់គ្រូ៖} សូមពិចារណាប្រើតារាងតម្លៃខ្ទង់ និងបន្ទាត់ចំនួន។ លើកទឹកចិត្តឱ្យសិស្សប្រើវិធីសាស្ត្រចម្រុះ (ឧ. លក្ខណៈបំបែក) និងការគណនាដោយប្រើខួរក្បាល។

\end{document}
