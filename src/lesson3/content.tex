\clearpage
\setcounter{section}{0}

\section*{មេរៀនទី៣៖ ប្រភាគ ពហុគុណ ចំនួនបឋម និងចំនួនមិនបឋម}

\subsection*{វត្ថុបំណងនៃការសិក្សា}
នៅចុងបញ្ចប់នៃមេរៀននេះ សិស្សនឹងអាច៖
\begin{itemize}[label=-]
    \item កំណត់និយមន័យនៃប្រភាគ, ពហុគុណ, ចំនួនបឋម, និងចំនួនមិនបឋម (ចំនួនចំរុះ)។
    \item អនុវត្តប្រមាណវិធីបូក ដក គុណ និងចែកជាមួយប្រភាគ។
    \item រកពហុគុណនៃចំនួនមួយ។
    \item សម្គាល់រវាងចំនួនបឋម និងចំនួនមិនបឋម (ចំនួនចំរុះ)។
\end{itemize}

\section{គោលគំនិតសំខាន់ៗ}

\subsection{ប្រភាគ ($\mathbb{Q}$)}
ប្រភាគតំណាងឱ្យចំណែកមួយនៃចំនួនទាំងមូល។ វាត្រូវបានសរសេរក្នុងទម្រង់ $\frac{a}{b}$ ដែល $a$ ជាភាគយក ហើយ $b$ ជាភាគបែង ($b \neq 0$)។

\subsubsection{ប្រមាណវិធីលើប្រភាគ}
\begin{itemize}[label=-]
    \item \textbf{ការបូក និងដក៖} ដើម្បីបូកឬដកប្រភាគ, ត្រូវតម្រូវឱ្យមានភាគបែងរួម។ $\frac{a}{b} \pm \frac{c}{d} = \frac{ad \pm bc}{bd}$។\\
    ឧទាហរណ៍៖ $\frac{1}{2} + \frac{1}{3} = \frac{1 \times 3 + 1 \times 2}{2 \times 3} = \frac{3+2}{6} = \frac{5}{6}$។
    \item \textbf{ការគុណ៖} គុណភាគយកនឹងភាគយក និងភាគបែងនឹងភាគបែង។ $\frac{a}{b} \times \frac{c}{d} = \frac{ac}{bd}$។\\
    ឧទាហរណ៍៖ $\frac{2}{3} \times \frac{4}{5} = \frac{2 \times 4}{3 \times 5} = \frac{8}{15}$។
    \item \textbf{ការចែក៖} គុណប្រភាគទីមួយនឹងចម្រាស់នៃប្រភាគទីពីរ។ $\frac{a}{b} \div \frac{c}{d} = \frac{a}{b} \times \frac{d}{c} = \frac{ad}{bc}$។\\
    ឧទាហរណ៍៖ $\frac{1}{2} \div \frac{1}{4} = \frac{1}{2} \times \frac{4}{1} = \frac{4}{2} = 2$។
\end{itemize}

\subsection{ពហុគុណ}
ពហុគុណនៃចំនួនមួយគឺជាលទ្ធផលដែលទទួលបានពីការគុណចំនួននោះជាមួយចំនួនគត់ធម្មជាតិ។\\
ឧទាហរណ៍៖ ពហុគុណនៃ 4 គឺ $4, 8, 12, 16, \dots$ (ព្រោះ $4\times1=4, 4\times2=8, \dots$)។

\subsection{ចំនួនបឋម}
ចំនួនបឋមគឺជាចំនួនគត់ធម្មជាតិធំជាង 1 ដែលមានតួចែកតែពីរគត់គឺ 1 និងខ្លួនវាផ្ទាល់។\\
ឧទាហរណ៍៖ $2, 3, 5, 7, 11, 13, \dots$។

\subsection{ចំនួនមិនបឋម (ចំនួនចំរុះ)}
ចំនួនមិនបឋម (ឬចំនួនចំរុះ) គឺជាចំនួនគត់ធម្មជាតិធំជាង 1 ដែលមិនមែនជាចំនួនបឋម។ វាមានតួចែកច្រើនជាងពីរ។\\
ឧទាហរណ៍៖ $4, 6, 8, 9, 10, 12, \dots$។

\section{ឧទាហរណ៍អនុវត្ត}

\begin{example}{ការគណនាប្រភាគ}
    គណនា៖ $\frac{3}{4} + \frac{1}{6}$។
    \begin{solution}
        រកភាគបែងរួមតូចបំផុត (LCM) នៃ 4 និង 6 គឺ 12។\\
        $\frac{3}{4} + \frac{1}{6} = \frac{3 \times 3}{12} + \frac{1 \times 2}{12} = \frac{9+2}{12} = \frac{11}{12}$។
    \end{solution}
\end{example}

\begin{example}{ការកំណត់ចំនួនបឋម}
    តើ 17 ជាចំនួនបឋមឬទេ?
    \begin{solution}
        បាទ, 17 ជាចំនួនបឋម ព្រោះវាអាចចែកដាច់តែនឹង 1 និង 17 ប៉ុណ្ណោះ។
    \end{solution}
\end{example}

\section{លំហាត់អនុវត្ត}
\begin{enumerate}[label=\arabic*.]
    \item គណនា៖ $\frac{2}{5} + \frac{3}{10}$។
    \item គណនា៖ $\frac{5}{6} - \frac{1}{4}$។
    \item គណនា៖ $\frac{3}{7} \times \frac{2}{5}$។
    \item រកពហុគុណ 5 ដំបូងនៃ 6។
    \item តើ 29 ជាចំនួនបឋម ឬចំនួនចំរុះ?
    \item រាយតួចែកទាំងអស់នៃ 18។
\end{enumerate}

\section{ចម្លើយលំហាត់អនុវត្ត}
\begin{enumerate}[label=\arabic*.]
    \item $\frac{2}{5} + \frac{3}{10} = \frac{4}{10} + \frac{3}{10} = \frac{7}{10}$
    \item $\frac{5}{6} - \frac{1}{4} = \frac{10}{12} - \frac{3}{12} = \frac{7}{12}$
    \item $\frac{3}{7} \times \frac{2}{5} = \frac{6}{35}$
    \item ពហុគុណ 5 ដំបូងនៃ 6 គឺ $6, 12, 18, 24, 30$។
    \item 29 ជាចំនួនបឋម។
    \item តួចែកនៃ 18 គឺ $1, 2, 3, 6, 9, 18$។
\end{enumerate}
